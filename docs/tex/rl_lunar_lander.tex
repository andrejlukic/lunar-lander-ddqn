\documentclass{article}

% if you need to pass options to natbib, use, e.g.:
% \PassOptionsToPackage{numbers, compress}{natbib}
% before loading rl_project.

% to compile a camera-ready version, add the [final] option, e.g.:
 \usepackage[final]{rl_project}

% to avoid loading the natbib package, add option nonatbib:
% \usepackage[nonatbib]{rl_project}

\usepackage[utf8]{inputenc} % allow utf-8 input
\usepackage[T1]{fontenc}    % use 8-bit T1 fonts
\usepackage{hyperref}       % hyperlinks
\usepackage{url}            % simple URL typesetting
\usepackage{booktabs}       % professional-quality tables
\usepackage{amsfonts}       % blackboard math symbols
\usepackage{nicefrac}       % compact symbols for 1/2, etc.
\usepackage{microtype}      % microtypography
\usepackage{graphicx}


% Give your project report an appropriate title!

\title{RL Project Template}


% The \author macro works with any number of authors. There are two
% commands used to separate the names and addresses of multiple
% authors: \And and \AND.
%
% Using \And between authors leaves it to LaTeX to determine where to
% break the lines. Using \AND forces a line break at that point. So,
% if LaTeX puts 3 of 4 authors names on the first line, and the last
% on the second line, try using \AND instead of \And before the third
% author name.

\author{
  Andrej Lukic
  \\
  Department of Computer Science\\
  University of Bath\\
  Bath, BA2 7AY \\
  \texttt{al2274@bath.ac.uk} \\
  %% examples of more authors
  %% \And
  %% Coauthor \\
  %% Affiliation \\
  %% Address \\
  %% \texttt{email} \\
  %% \AND
  %% Coauthor \\
  %% Affiliation \\
  %% Address \\
  %% \texttt{email} \\
  %% \And
  %% Coauthor \\
  %% Affiliation \\
  %% Address \\
  %% \texttt{email} \\
  %% \And
  %% Coauthor \\
  %% Affiliation \\
  %% Address \\
  %% \texttt{email} \\
}

\begin{document}

\maketitle

\section{Problem Definition}
The problem chosen for this assignment is the Lunar Lander problem because landing a small rocket on the moon is a bit correlated with my interest in flight controllers used in custom made drones. The environment for testing the algorithm is freely available on the \href{https://gymnasium.farama.org}{Gymnasium} web site (it's an actively maintained fork of the original \href{https://github.com/openai/gym}{OpenAI Gym} developed by Oleg Klimov.

The \href{https://gymnasium.farama.org/environments/box2d/lunar_lander/}{Lunar Lander}  is a classic rocket trajectory optimisation problem (\href{https://gymnasium.farama.org/environments/box2d/lunar_lander/}{comprehensive environment description is found on the gymnasium web site} ). In the simulation, the spacecraft has a main engine and two lateral boosters that can be used to control its descent and the orientation of the spacecraft. The spacecraft is subject to the moon's gravitational pull, and the engines have an unlimited amount of fuel. The spacecraft must navigate to the landing spot between two flags at coordinates (0,0) without crashing. Landing outside of the landing pad is possible. The lander starts at the top center of the viewport with a random initial force applied to its center of mass. The environment has 4 discrete actions: 

\begin{itemize}
  \item 0: do nothing
  \item 1: fire left orientation engine
  \item 2: fire main engine
  \item 3: fire right orientation engine
\end{itemize}

The state is an 8-dimensional vector: the coordinates of the lander in x \& y, its linear velocities in x \& y, its angle, its angular velocity, and two booleans that represent whether each leg is in contact with the ground or not.

After every step a reward is granted. The total reward of an episode is the sum of the rewards for all the steps within that episode.

For each step, the reward:
\begin{itemize}
  \item is increased/decreased the closer/further the lander is to the landing pad.
  \item is increased/decreased the slower/faster the lander is moving.
  \item is decreased the more the lander is tilted (angle not horizontal).
  \item is increased by 10 points for each leg that is in contact with the ground.
  \item is decreased by 0.03 points each frame a side engine is firing.
  \item is decreased by 0.3 points each frame the main engine is firing.
\end{itemize}

The episode receive an additional reward of -100 or +100 points for crashing or landing safely respectively. An episode is considered a solution if it scores at least 200 points.

The episode finishes if:
\begin{itemize}
  \item the lander crashes (the lander body gets in contact with the moon);
  \item the lander gets outside of the viewport (x coordinate is greater than 1);
  \item the lander is not awake. From the Box2D docs, a body which is not awake is a body which doesn’t move and doesn’t collide with any other body:
\end{itemize}

\section{Background}

\section{Method}

\section{Results}

\section{Discussion}

\section{Future Work}

\section{Personal Experience}


\section*{References}
\small

\normalsize
\newpage
\section*{Appendices}
If you have additional content that you would like to include in the appendices, please do so here.
There is no limit to the length of your appendices, but we are not obliged to read them in their entirety while marking. The main body of your report should contain all essential information, and content in the appendices should be clearly referenced where it's needed elsewhere.
\subsection*{Appendix A: Example Appendix 1}
\subsection*{Appendix B: Example Appendix 2}

\end{document}
